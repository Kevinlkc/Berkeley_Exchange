\documentclass[]{article}
\usepackage{lmodern}
\usepackage{amssymb,amsmath}
\usepackage{ifxetex,ifluatex}
\usepackage{fixltx2e} % provides \textsubscript
\ifnum 0\ifxetex 1\fi\ifluatex 1\fi=0 % if pdftex
  \usepackage[T1]{fontenc}
  \usepackage[utf8]{inputenc}
\else % if luatex or xelatex
  \ifxetex
    \usepackage{mathspec}
  \else
    \usepackage{fontspec}
  \fi
  \defaultfontfeatures{Ligatures=TeX,Scale=MatchLowercase}
\fi
% use upquote if available, for straight quotes in verbatim environments
\IfFileExists{upquote.sty}{\usepackage{upquote}}{}
% use microtype if available
\IfFileExists{microtype.sty}{%
\usepackage{microtype}
\UseMicrotypeSet[protrusion]{basicmath} % disable protrusion for tt fonts
}{}
\usepackage[margin=1in]{geometry}
\usepackage{hyperref}
\hypersetup{unicode=true,
            pdftitle={Hw3},
            pdfauthor={Kevin Luo},
            pdfborder={0 0 0},
            breaklinks=true}
\urlstyle{same}  % don't use monospace font for urls
\usepackage{color}
\usepackage{fancyvrb}
\newcommand{\VerbBar}{|}
\newcommand{\VERB}{\Verb[commandchars=\\\{\}]}
\DefineVerbatimEnvironment{Highlighting}{Verbatim}{commandchars=\\\{\}}
% Add ',fontsize=\small' for more characters per line
\usepackage{framed}
\definecolor{shadecolor}{RGB}{248,248,248}
\newenvironment{Shaded}{\begin{snugshade}}{\end{snugshade}}
\newcommand{\KeywordTok}[1]{\textcolor[rgb]{0.13,0.29,0.53}{\textbf{#1}}}
\newcommand{\DataTypeTok}[1]{\textcolor[rgb]{0.13,0.29,0.53}{#1}}
\newcommand{\DecValTok}[1]{\textcolor[rgb]{0.00,0.00,0.81}{#1}}
\newcommand{\BaseNTok}[1]{\textcolor[rgb]{0.00,0.00,0.81}{#1}}
\newcommand{\FloatTok}[1]{\textcolor[rgb]{0.00,0.00,0.81}{#1}}
\newcommand{\ConstantTok}[1]{\textcolor[rgb]{0.00,0.00,0.00}{#1}}
\newcommand{\CharTok}[1]{\textcolor[rgb]{0.31,0.60,0.02}{#1}}
\newcommand{\SpecialCharTok}[1]{\textcolor[rgb]{0.00,0.00,0.00}{#1}}
\newcommand{\StringTok}[1]{\textcolor[rgb]{0.31,0.60,0.02}{#1}}
\newcommand{\VerbatimStringTok}[1]{\textcolor[rgb]{0.31,0.60,0.02}{#1}}
\newcommand{\SpecialStringTok}[1]{\textcolor[rgb]{0.31,0.60,0.02}{#1}}
\newcommand{\ImportTok}[1]{#1}
\newcommand{\CommentTok}[1]{\textcolor[rgb]{0.56,0.35,0.01}{\textit{#1}}}
\newcommand{\DocumentationTok}[1]{\textcolor[rgb]{0.56,0.35,0.01}{\textbf{\textit{#1}}}}
\newcommand{\AnnotationTok}[1]{\textcolor[rgb]{0.56,0.35,0.01}{\textbf{\textit{#1}}}}
\newcommand{\CommentVarTok}[1]{\textcolor[rgb]{0.56,0.35,0.01}{\textbf{\textit{#1}}}}
\newcommand{\OtherTok}[1]{\textcolor[rgb]{0.56,0.35,0.01}{#1}}
\newcommand{\FunctionTok}[1]{\textcolor[rgb]{0.00,0.00,0.00}{#1}}
\newcommand{\VariableTok}[1]{\textcolor[rgb]{0.00,0.00,0.00}{#1}}
\newcommand{\ControlFlowTok}[1]{\textcolor[rgb]{0.13,0.29,0.53}{\textbf{#1}}}
\newcommand{\OperatorTok}[1]{\textcolor[rgb]{0.81,0.36,0.00}{\textbf{#1}}}
\newcommand{\BuiltInTok}[1]{#1}
\newcommand{\ExtensionTok}[1]{#1}
\newcommand{\PreprocessorTok}[1]{\textcolor[rgb]{0.56,0.35,0.01}{\textit{#1}}}
\newcommand{\AttributeTok}[1]{\textcolor[rgb]{0.77,0.63,0.00}{#1}}
\newcommand{\RegionMarkerTok}[1]{#1}
\newcommand{\InformationTok}[1]{\textcolor[rgb]{0.56,0.35,0.01}{\textbf{\textit{#1}}}}
\newcommand{\WarningTok}[1]{\textcolor[rgb]{0.56,0.35,0.01}{\textbf{\textit{#1}}}}
\newcommand{\AlertTok}[1]{\textcolor[rgb]{0.94,0.16,0.16}{#1}}
\newcommand{\ErrorTok}[1]{\textcolor[rgb]{0.64,0.00,0.00}{\textbf{#1}}}
\newcommand{\NormalTok}[1]{#1}
\usepackage{graphicx,grffile}
\makeatletter
\def\maxwidth{\ifdim\Gin@nat@width>\linewidth\linewidth\else\Gin@nat@width\fi}
\def\maxheight{\ifdim\Gin@nat@height>\textheight\textheight\else\Gin@nat@height\fi}
\makeatother
% Scale images if necessary, so that they will not overflow the page
% margins by default, and it is still possible to overwrite the defaults
% using explicit options in \includegraphics[width, height, ...]{}
\setkeys{Gin}{width=\maxwidth,height=\maxheight,keepaspectratio}
\IfFileExists{parskip.sty}{%
\usepackage{parskip}
}{% else
\setlength{\parindent}{0pt}
\setlength{\parskip}{6pt plus 2pt minus 1pt}
}
\setlength{\emergencystretch}{3em}  % prevent overfull lines
\providecommand{\tightlist}{%
  \setlength{\itemsep}{0pt}\setlength{\parskip}{0pt}}
\setcounter{secnumdepth}{0}
% Redefines (sub)paragraphs to behave more like sections
\ifx\paragraph\undefined\else
\let\oldparagraph\paragraph
\renewcommand{\paragraph}[1]{\oldparagraph{#1}\mbox{}}
\fi
\ifx\subparagraph\undefined\else
\let\oldsubparagraph\subparagraph
\renewcommand{\subparagraph}[1]{\oldsubparagraph{#1}\mbox{}}
\fi

%%% Use protect on footnotes to avoid problems with footnotes in titles
\let\rmarkdownfootnote\footnote%
\def\footnote{\protect\rmarkdownfootnote}

%%% Change title format to be more compact
\usepackage{titling}

% Create subtitle command for use in maketitle
\providecommand{\subtitle}[1]{
  \posttitle{
    \begin{center}\large#1\end{center}
    }
}

\setlength{\droptitle}{-2em}

  \title{Hw3}
    \pretitle{\vspace{\droptitle}\centering\huge}
  \posttitle{\par}
    \author{Kevin Luo}
    \preauthor{\centering\large\emph}
  \postauthor{\par}
      \predate{\centering\large\emph}
  \postdate{\par}
    \date{2019/9/22}


\begin{document}
\maketitle

\begin{Shaded}
\begin{Highlighting}[]
\NormalTok{A1 <-}\StringTok{ }\KeywordTok{diag}\NormalTok{(}\KeywordTok{c}\NormalTok{(}\DecValTok{1}\NormalTok{,}\DecValTok{2}\NormalTok{,}\DecValTok{2}\NormalTok{), }\DataTypeTok{nrow =} \DecValTok{3}\NormalTok{)}
\NormalTok{A2 <-}\StringTok{ }\KeywordTok{diag}\NormalTok{(}\KeywordTok{c}\NormalTok{(}\DecValTok{1}\NormalTok{,}\DecValTok{2}\NormalTok{,}\DecValTok{0}\NormalTok{), }\DataTypeTok{nrow =} \DecValTok{3}\NormalTok{)}
\NormalTok{b <-}\StringTok{ }\KeywordTok{matrix}\NormalTok{(}\KeywordTok{c}\NormalTok{(}\DecValTok{1}\NormalTok{,}\DecValTok{1}\NormalTok{,}\DecValTok{0}\NormalTok{), }\DataTypeTok{ncol =} \DecValTok{1}\NormalTok{)}
\NormalTok{epsilon <-}\StringTok{ }\KeywordTok{c}\NormalTok{(}\FloatTok{1e-8}\NormalTok{,}\FloatTok{1e-8}\NormalTok{,}\FloatTok{1e-8}\NormalTok{)}
\NormalTok{lambda <-}\StringTok{ }\FloatTok{0.1}
\NormalTok{converge <-}\StringTok{ }\ControlFlowTok{function}\NormalTok{(X, lastX)\{}
  \ControlFlowTok{for}\NormalTok{ (i }\ControlFlowTok{in} \DecValTok{1}\OperatorTok{:}\KeywordTok{length}\NormalTok{(X))\{}
    \ControlFlowTok{if}\NormalTok{ (}\KeywordTok{abs}\NormalTok{(X[i]}\OperatorTok{-}\NormalTok{lastX[i])}\OperatorTok{>}\NormalTok{epsilon)\{}
      \KeywordTok{return}\NormalTok{(}\OtherTok{FALSE}\NormalTok{)}
\NormalTok{    \}}
\NormalTok{  \}}
  \KeywordTok{return}\NormalTok{(}\OtherTok{TRUE}\NormalTok{)}
\NormalTok{\}}
\ControlFlowTok{for}\NormalTok{ (i }\ControlFlowTok{in} \DecValTok{1}\OperatorTok{:}\DecValTok{5}\NormalTok{)\{}
\NormalTok{  X =}\StringTok{ }\KeywordTok{rnorm}\NormalTok{(}\DecValTok{3}\NormalTok{)}
\NormalTok{  lastX <-}\StringTok{ }\NormalTok{X}\OperatorTok{+}\DecValTok{1}
  \ControlFlowTok{while}\NormalTok{ (}\KeywordTok{converge}\NormalTok{(X, lastX) }\OperatorTok{==}\StringTok{ }\OtherTok{FALSE}\NormalTok{)\{}
\NormalTok{    lastX =}\StringTok{ }\NormalTok{X}
\NormalTok{    X =}\StringTok{ }\NormalTok{X }\OperatorTok{-}\StringTok{ }\NormalTok{lambda}\OperatorTok{*}\NormalTok{(A1 }\OperatorTok\StringTok{ }\NormalTok{X }\OperatorTok{-}\StringTok{ }\NormalTok{b)}
\NormalTok{  \}}
  \KeywordTok{print}\NormalTok{(X)}
\NormalTok{\}}
\end{Highlighting}
\end{Shaded}

\begin{verbatim}
##               [,1]
## [1,]  9.999999e-01
## [2,]  5.000000e-01
## [3,] -3.013299e-16
##               [,1]
## [1,]  9.999999e-01
## [2,]  5.000000e-01
## [3,] -3.698993e-15
##               [,1]
## [1,]  9.999999e-01
## [2,]  5.000000e-01
## [3,] -1.410955e-15
##               [,1]
## [1,]  1.000000e+00
## [2,]  5.000000e-01
## [3,] -5.005869e-14
##               [,1]
## [1,]  9.999999e-01
## [2,]  5.000000e-01
## [3,] -2.037503e-17
\end{verbatim}

\begin{Shaded}
\begin{Highlighting}[]
\CommentTok{# Compare the results to $x* = A1^\{-1\}b$}
\KeywordTok{solve}\NormalTok{(A1) }\OperatorTok\NormalTok{b}
\end{Highlighting}
\end{Shaded}

\begin{verbatim}
##      [,1]
## [1,]  1.0
## [2,]  0.5
## [3,]  0.0
\end{verbatim}

\begin{Shaded}
\begin{Highlighting}[]
\CommentTok{# They converged to the same x*}
\end{Highlighting}
\end{Shaded}

But that is not the case when A is not invertible. Mathematically, the
optimization problem has infinite set of solutions. The converged result
will thus depends on the initialization process.

\begin{Shaded}
\begin{Highlighting}[]
\ControlFlowTok{for}\NormalTok{ (i }\ControlFlowTok{in} \DecValTok{1}\OperatorTok{:}\DecValTok{5}\NormalTok{)\{}
\NormalTok{  X =}\StringTok{ }\KeywordTok{rnorm}\NormalTok{(}\DecValTok{3}\NormalTok{)}
\NormalTok{  lastX <-}\StringTok{ }\NormalTok{X}\OperatorTok{+}\DecValTok{1}
  \ControlFlowTok{while}\NormalTok{ (}\KeywordTok{converge}\NormalTok{(X, lastX) }\OperatorTok{==}\StringTok{ }\OtherTok{FALSE}\NormalTok{)\{}
\NormalTok{    lastX =}\StringTok{ }\NormalTok{X}
\NormalTok{    X =}\StringTok{ }\NormalTok{X }\OperatorTok{-}\StringTok{ }\NormalTok{lambda}\OperatorTok{*}\NormalTok{(A2 }\OperatorTok\StringTok{ }\NormalTok{X }\OperatorTok{-}\StringTok{ }\NormalTok{b)}
\NormalTok{  \}}
  \KeywordTok{print}\NormalTok{(X)}
\NormalTok{\}}
\end{Highlighting}
\end{Shaded}

\begin{verbatim}
##           [,1]
## [1,] 0.9999999
## [2,] 0.5000000
## [3,] 1.1574709
##            [,1]
## [1,]  0.9999999
## [2,]  0.5000000
## [3,] -0.8228044
##            [,1]
## [1,]  0.9999999
## [2,]  0.5000000
## [3,] -0.3215081
##            [,1]
## [1,]  0.9999999
## [2,]  0.5000000
## [3,] -1.2682850
##           [,1]
## [1,] 0.9999999
## [2,] 0.5000000
## [3,] 1.5508908
\end{verbatim}


\end{document}
